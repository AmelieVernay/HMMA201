\makeatletter
%--------------------------------------------------------------------------------

%\usepackage[french]{babel}
\usepackage[a4paper,hmargin=20mm,vmargin=25mm]{geometry}
\usepackage{dsfont}
\usepackage[utf8]{inputenc}
\usepackage[T1]{fontenc}
\usepackage{lmodern}

%-------------------------------- AMS --------------------------------------%
\usepackage{amsmath}
\usepackage{amsbsy}
\usepackage{amsfonts}
\usepackage{amssymb}
\usepackage{amscd}
\usepackage{amsthm}


\theoremstyle{definition}
%\newtheorem{defi}{Definition}[section]
\newtheorem{defi}{Definition}[]
\newtheorem{remark}[defi]{Remarque}
\newtheorem{prop}[]{Proposition}
\newtheorem{theo}[]{Th\'eor\`eme}



\usepackage{xcolor}
\usepackage{mathtools}
\usepackage{eurosym}
\usepackage{nicefrac}
\usepackage{latexsym}

\usepackage{multicol}
\usepackage[inline]{enumitem}
\setlist{nosep}
\setlist[itemize,1]{,label=$-$}

\usepackage{sectsty}
%\sectionfont{}
\allsectionsfont{\normalfont\sffamily\bfseries\normalsize}

\usepackage{graphicx}
\usepackage{tikz}
\usetikzlibrary{calc,fadings,decorations.pathreplacing,matrix,arrows,decorations.text}
 \usetikzlibrary{patterns}

\newcommand\chideux[1]{#1<=0||(#1!=int(#1))?1/0:x<=0?0.0:exp((0.5*#1-1.0)*log(x)-0.5*x-lgamma(0.5*#1)-#1*0.5*log(2))}
\newcommand\gauss[2]{1/(#2*sqrt(2*pi))*exp(-((x-#1)^2)/(2*#2^2))} 
\newcommand\student[1]{gamma(.5*(#1+1))/(sqrt(#1*pi)*gamma(.5*#1))*((1+x^2/#1)^(-.5*(#1+1)))}

\usepackage[framemethod=tikz]{mdframed}
\mdfdefinestyle{commandline}{leftmargin=.07\linewidth, rightmargin=.07\linewidth,backgroundcolor=gray!20,linewidth=0pt}
\mdfdefinestyle{codeline}{leftmargin=.01\linewidth, rightmargin=.01\linewidth,linewidth=1pt,backgroundcolor=gray!20,linewidth=0pt}


\usepackage{textcomp}
\usepackage{listings}
\lstloadlanguages{R}
\lstset{upquote=true,
	%columns=flexible,
	keepspaces=true,
	breaklines=true,
	breakindent=0pt,
	basicstyle={\small\ttfamily}
	}
\lstset{extendedchars=true,
	literate={é}{{\'e}}1
		 {è}{{\`e}}1
		 {à}{{\`a}}1
		 {ç}{{\c{c}}}1
		 {œ}{{\oe}}1
		 {ù}{{\`u}}1
		 {É}{{\'E}}1
		 {È}{{\`E}}1
		 {À}{{\`A}}1
		 {Ç}{{\c{C}}}1
		 {Œ}{{\OE}}1
		 {Ê}{{\^E}}1
		 {ê}{{\^e}}1
		 {î}{{\^i}}1
		 {ô}{{\^o}}1
		 {û}{{\^u}}1
		 {°}{{\degre}}1
	}

\lstnewenvironment{script}{%
\lstset{language=,
	    aboveskip=0pt,
	belowskip=0pt
	}
}{}
\surroundwithmdframed[style=commandline]{script}

\lstdefinestyle{codeR}{%
	aboveskip=0pt,
	belowskip=0pt,
	%frame=single,
	commentstyle=\color{green!50!black}\itshape,
	language=R,
	keywordstyle=\color{violet!75},
	stringstyle=\color{red},
	showstringspaces=false,
      %  emph={output,input}, emphstyle=\color{brown!75},
      %  emph={maFonction}, emphstyle=\underline,
	}

\lstnewenvironment{codeR}[1][]{\lstset{style=codeR,#1}}{}
\surroundwithmdframed[style=codeline]{codeR}


\usepackage{algorithm}
\usepackage{algpseudocode}


\usepackage{pgfplots}
\usepgfplotslibrary{fillbetween}
\pgfplotsset{compat=newest}
%\usepgfplotslibrary{external} 
%\tikzexternalize[prefix=./output_latex/]
%\DeclareSymbolFont{RalphSmithFonts}{U}{rsfs}{m}{n}
%\DeclareSymbolFontAlphabet{\mathscr}{RalphSmithFonts}
%\def\mathcal#1{{\mathscr #1}}



% ------------------------------------------- Command -------------------------------%
\providecommand{\abs}[1]{\left|#1\right|}
\providecommand{\norm}[1]{\left\Vert#1\right\Vert}
\providecommand{\U}{\mathcal{U}}
\providecommand{\R}{\mathbb{R}}
\providecommand{\Cc}{\mathcal{C}}
\providecommand{\reg}[1]{\mathcal{C}^{#1}}
\providecommand{\1}{\mathds{1}}
\providecommand{\N}{\mathbb{N}}
\providecommand{\Z}{\mathbb{Z}}
\providecommand{\C}{\mathbb{C}}
\providecommand{\F}{\mathbb{F}}
\providecommand{\K}{\mathbb{K}}
\providecommand{\p}{\partial}
\providecommand{\gR}{\textsc{R}}
\providecommand{\one}{\mathds{1}}
%\renewcommand{\E}{\mathbb{E}}
\renewcommand{\P}{\mathbb{P}}
\providecommand{\ie}{\textit{i.e. }}

\newcommand\rst[2]{{#1}_{\restriction_{#2}}}
%Operateur
\providecommand{\abs}[1]{\left\lvert#1\right\rvert}
\providecommand{\sabs}[1]{\lvert#1\rvert}
\providecommand{\babs}[1]{\bigg\lvert#1\bigg\rvert}
\providecommand{\norm}[1]{\left\lVert#1\right\rVert}
\providecommand{\bnorm}[1]{\bigg\lVert#1\bigg\rVert}
\providecommand{\snorm}[1]{\lVert#1\rVert}
\providecommand{\prs}[1]{\left\langle #1\right\rangle}
\providecommand{\sprs}[1]{\langle #1\rangle}
\providecommand{\bprs}[1]{\bigg\langle #1\bigg\rangle}

\DeclareMathOperator{\deet}{Det}
\DeclareMathOperator{\supp}{Supp}
\DeclareMathOperator{\sinc}{sinc}


\usepackage{ifthen}

\newcommand{\eno}[1]{%
	\ifthenelse{\equal{\version}{eno}}{#1}{}%
}
\newcommand{\cor}[1]{%
        \ifthenelse{\equal{\version}{cor}}{
\medskip 

{\small \color{gray} #1}

\medskip 
}{}
}

%------------------------------------------------------------------------------
\DeclareUnicodeCharacter{00A0}{~}
\makeatother

